% Appendix A

\chapter{Performance Metrics} % Main appendix title

\label{AppendixA} % For referencing this appendix elsewhere, use \ref{AppendixA}

\section{STOI}
\textit{\textbf{Short-Time Objective Intelligibility (STOI)}, measures the correlation between the short-time temporal envelopes of a reference (clean) audio signal and a degraded audio signal for speech intelligibility of human speech \cite{ref:ieee_stoi}. The value range of STOI is typically between 0 and 1, 0 being the worst and 1 being the best intelligibility. STOI values can also be considered to be percentage correct.}

\section{PESQ}
\textit{\textbf{Perceptual Evaluation of Speech Quality (PESQ)} is the standard metric recommended by the International Telecommunication Union (ITU) for analysing quality of a degraded signal with respect to a clean reference signal \cite{ref:pesq}. PESQ applies an auditory transform to produce a loudness spectrum, and compares the loudness spectra of a clean reference signal and a degraded signal to produce a score in a range of negative 0.5 to 4.5. This score is regarded to be a Mean Opinion Score (MOS). MOS can further be transformed in terms of listening objectivity metrics on a scale from 0 to 5 known as Listening Quality Objectivity (LQO). LQO scale is important as it is a mapping from MOS as per the human auditory response.}
